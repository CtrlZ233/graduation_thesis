%%==================================================
%% chapter04.tex for BIT Master Thesis
%% modified by yang yating
%% version: 0.1
%% last update: Dec 25th, 2016
%%==================================================
\chapter{ReL4系统实现}
\label{chap:ReL4_impl}

为了简洁高效地实现异步微内核的原型系统,本项目使用Rust语言在RISC-V平台上实现了一个兼容seL4基本系统调用的微内核ReL4,目前已经支持SMP架构和fast-path优化。在兼容seL4原始功能的基础(包括SMP和fast-path优化)上, ReL4实现了U-notification以及异步IPC和异步系统调用。在实现过程中对内核接口更改和使用的一些重要细节将在本章描述。

\section{新增系统调用}

\begin{table}
    \centering
    \begin{tabular}{|c|c|c|}
        \hline 
        syscall & 参数 & 描述 \\
        \hline
        UNotificationRegisterSender & ntfn\_cap & 注册通知发送端 \ \\
        \hline
        UNotificationRegisterReceiver & ntfn\_cap & 注册通知接收端 \ \\
        \hline
        RegisterAsyncSyscall & ntfn\_cap, buffer\_cap & 注册异步系统调用处理协程 \ \\
        \hline
        WakeAsyncSyscallHandler & - & 唤醒系统调用处理协程 \ \\
        \hline
    \end{tabular}
    \caption{ReL4中的新增系统调用}
    \label{tab:new_syscall}
\end{table}

如\ref{tab:new_syscall}所示,为了支持内核对U-notification的硬件资源管理,ReL4新增了系统调用:UNotificationRegisterSender 和 UNotificationRegisterReceiver 用于申请相关的硬件资源,其参数是notification内核对象对应的Capbability。内核会从UINTC和TAIC中分配对应的硬件资源索引,并将对应的索引绑定到TCB中,资源的释放不需要额外的系统调用,当TCB或notification对象销毁时会自动释放掉硬件资源。

此外,为了支持异步系统调用,共享缓冲区也需要通过系统调用(RegisterAsyncSyscall)注册给内核,内核会将共享缓冲区与TCB绑定,并为该线程注册处理该缓冲区请求的内核协程。最后,由于用户态中断不支持用户态直接通知内核态(TAIC支持),内核提供一个用于唤醒系统调用处理协程的系统调用WakeAsyncSyscallHandler,用户态根据硬件配置,如果不支持直接通知,则调用WakeAsyncSyscallHandler系统调用,否则直接发送中断将内核中对应的处理协程唤醒,并找到一个空闲的CPU核心发送核间中断去抢占执行。

为了对seL4进行兼容,这些系统调用均由异步运行时在初始化时进行调用,用户程序无需感知。

\section{异步IPC}

异步IPC作为ReL4中的主要的IPC方式,其实现依赖于异步运行时和U-notification。以IPC中最常见的Call为例,如\ref{fig:async_ipc}所示,客户端进程和服务端进程在双方建立连接时,首先会向内核申请两条通道所占用的硬件资源,然后每个进程中的runtime会注册一个dispatcher协程并将绑定0号中断向量,用于在TAIC资源不够的情况下,用软件方式唤醒worker协程。服务端和客户端的worker协程则负责发起和处理请求,由runtime将数据写入共享缓冲区中。由于中断向量有限,runtime会尝试为每个worker协程分配中断向量,分配到中断向量的协程,其唤醒过程由TAIC完成,没有分配到中断向量的协程,其唤醒过程由dispatcher协程完成。

\begin{figure*}[htbp]
    \centering
    \includegraphics[width=1.0\textwidth]{figures/async_ipc.drawio.pdf}
    \caption{异步IPC流程示意图}\label{fig:async_ipc}
  \end{figure*}

Call的主要流程分为以下几个阶段:
\begin{enumerate}
    \item 客户端发起请求:如\ref{alg:async_ipc_call}所示,用户态程序将以worker协程的形式发起IPC请求,异步运行时首先会尝试分配一个中断向量给worker协程,如果没有分配到,则复用dispatcher协程的0号中断,然后根据请求的数据和协程的协程号、中断号生成 IPCItem ,写入请求的环形缓冲区中并将当前协程阻塞,然后检查缓冲区的 $req\_handler\_status$ 标志位,如果对方的dispatcher协程已经就绪,那客户端无需通知对方进程,对方进程的异步运行时会在某个时刻调度到dispatcher协程并处理请求。如果对方的dispatcher协程处于阻塞状态,则异步运行时会将$req\_handler\_status$ 标志位置位,并发送U-notification通知对方进程将dispatcher协程唤醒后重启调度。
    \item 服务端处理请求并写回响应:如\ref{alg:async_ipc_reply}所示,服务端的dispatcher协程会在合适的时机读取出请求并进行解码和处理,然后根据处理结果构造响应的 IPCItem 并写入响应的环形缓冲区中,如果中断号是0号协程,则runtime会检查缓冲区中的 $req\_handler\_status$  标志位后尝试唤醒客户端的 dispatcher协程,否则直接通过TAIC唤醒客户端的worker协程。如果缓冲区内容为空,dispatcher协程会将 $req\_handler\_status$  标志位置空,并将自己阻塞。
    \item 客户端处理响应:客户端的 dispatcher协程会在合适的时机重新被调度并唤醒没有分配到TAIC资源的worker协程,唤醒后的worker协程会从缓冲区中读取响应并释放缓冲区资源。
\end{enumerate}

\begin{algorithm}[H]
    \caption{发起异步IPC请求的伪代码}\label{alg:async_ipc_call}
    \SetKwFunction{Fn}{fn}
    \SetKwProg{Prog}{}{}{}
    \SetKw{KwAwait}{await}
    \SetKw{KwYield}{yield\_now}
    \SetKw{KwLet}{let}
    \SetKw{KwSome}{Some}
    \SetKw{KwErr}{Err}
    \SetKw{KwReturn}{return}
    // 发起异步IPC的worker协程 \\
    \Prog{\Fn{async ipc\_call(cap, msg\_info) $\rightarrow$ Result<IPCItem>}}{
        vec = get\_vec\_from\_pool()\;
        item = IPCItem::new(vec, current\_cid(), msg\_info)\;
        buffer = get\_buffer\_from\_cap(cap)\;
        // 写入IPC请求 \\
        buffer.req\_ring\_buffer.write(item)\;
        \If{buffer.req\_co\_status == false}{
            // 发送异步通知唤醒处理协程 \\
            buffer.req\_co\_status = true\;
            u\_notification\_signal(0)\;
        }
        // 阻塞当前协程,等待响应唤醒 \\
        \If{\KwLet \KwSome(reply) = \KwYield().\KwAwait}{
            \KwReturn \KwSome(reply)\;
        }
        \KwReturn \KwErr(())\;
    }
\end{algorithm}

\begin{algorithm}[H]
    \caption{处理异步IPC请求的伪代码}\label{alg:async_ipc_reply}
    \SetKwFunction{Fn}{fn}
    \SetKwProg{Prog}{}{}{}
    \SetKw{KwAwait}{await}
    \SetKw{KwYield}{yield\_now}
    \SetKw{KwLet}{let}
    \SetKw{KwSome}{Some}
    \SetKw{KwErr}{Err}
    \SetKw{KwReturn}{return}
    // 处理异步IPC请求的dispatcher协程 \\
    \Prog{\Fn{async ipc\_recv\_reply(cap)}}{
        buffer = get\_buffer\_from\_cap(cap)\;
        \While{true}{
            \If{\KwLet \KwSome(item) = buffer.req\_ring\_buffer.get()}{
                reply = handle\_item(item)\;
                // 写入IPC响应 \\
                buffer.resp\_ring\_buffer.write(reply)\;
                \If{buffer.reply\_co\_status == false}{
                    // 发送异步通知唤醒对端协程 \\
                    buffer.reply\_co\_status = true\;
                    u\_notification\_signal(item.vec)\;
                }
            }\Else{
                // 没有其他请求,阻塞自身并切换到其他可执行的协程 \\
                buffer.req\_co\_status = false\;
                \KwYield().\KwAwait\;
            }
        }
    }
\end{algorithm}


\section{异步系统调用}

从系统架构设计的角度来看,异步系统调用可以视为一类特殊的异步IPC通信过程,其特殊性主要体现在接收端为操作系统内核。为了支持这一特性,ReL4在内核层面实现了一套与用户态异步运行时相对应的内核态异步处理机制。通过对比分析异步系统调用与常规异步IPC的实现差异,我们发现以下两个关键区别需要特别处理:

首先,在通知机制方面,由于内核作为接收端运行在特权级,而用户态中断下的硬件架构仅支持用户态间的通知传递,这使得发送端无法直接使用U-notification来唤醒内核。针对这一限制,ReL4通过引入专门的唤醒系统调用(WakeAsyncSyscallHandler)来建立用户态与内核态之间的通知通道,既保持了接口的统一性,又确保了系统功能的完备性。

其次,在任务调度方面,内核环境具有更复杂的执行上下文。与用户态进程可以专注于处理异步IPC请求不同,内核还需要同时处理硬件中断、异常处理、线程调度等关键任务。这种多任务特性使得内核中的异步系统调用调度需要更精细的优先级管理策略。传统解决方案依赖时钟中断来触发异步请求处理,但这可能导致两个问题:1) 空闲CPU核心由于没有时钟中断触发而无法及时响应请求;2) 正在执行低优先级任务的CPU核心无法被及时抢占。为优化这一问题,ReL4采用了核间中断(IPI)机制来实现更灵活的调度唤醒,在保证系统实时性的同时提高CPU资源利用率。

在具体实现上,ReL4设计了一个分层的优先级调度框架来协调各类内核任务的执行顺序。该框架为每个CPU核心维护一个动态的执行优先级(exec\_prio),其取值反映了当前执行任务的紧急程度:

\begin{enumerate}
    \item 空闲状态(exec\_prio=256):当CPU核心运行空闲线程时处于最低优先级,可随时被更高优先级的异步请求抢占;
    \item 内核关键任务(exec\_prio=0):处理中断、异常等不可抢占的核心功能时处于最高优先级;
    \item 用户态任务(exec\_prio=thread\_priority):执行用户程序时采用对应线程的优先级,支持有限度的抢占。
\end{enumerate}

当用户态发起异步系统调用时,唤醒流程如算法\ref{alg:wake_syscall_handler}所示:首先通过系统调用陷入内核,然后调度器会检查当前系统状态,优先选择空闲CPU核心或可抢占的低优先级CPU核心(通过IPI)来立即处理请求。若无合适核心可用,则延迟到下次时钟中断处理。这种设计既保证了关键内核任务的优先执行,又最大限度地减少了异步系统调用的响应延迟,实现了安全性与性能的平衡。

\lstdefinelanguage{rust}{
    keywords={fn, if, let, Some, as, in},
    keywordstyle=\color{blue}\bfseries,
    ndkeywords={self, current, cid, cpu_id, exec_prio, mask},
    ndkeywordstyle=\color{magenta}\bfseries,
    identifierstyle=\color{black},
    sensitive=false,
    comment=[l]{//},
    morecomment=[s]{/*}{*/},
    commentstyle=\color{gray}\ttfamily,
    stringstyle=\color{red}\ttfamily,
    morestring=[b]',
    morestring=[b]"
}

\begin{algorithm}[H]
    \caption{唤醒内核中异步处理协程的伪代码}\label{alg:wake_syscall_handler}
    \SetKwFunction{Fn}{fn}
    \SetKwProg{Prog}{}{}{}
    \Prog{\Fn{wake\_syscall\_handler()}}{
        current = get\_current\_thread(); \\
        \If{let Some(cid) = current.async\_sys\_handler\_cid} {
            // 将处理协程加入就绪队列 \\
            coroutine\_wake(cid); \\
            current\_exec\_prio = current.tcb\_prio; \\
            // 获取CPU核心中最低的执行优先级 \\
            (cpu\_id, exec\_prio) = get\_max\_exec\_prio(); \\
            \If{current\_exec\_prio < exec\_prio} {
                // 抢占低执行优先级的核心 \\
                mask = 1 $\ll$ cpu\_id; \\
                ipi\_send\_mask(mask, ASYNC\_SYSCALL\_HANDLE); \\
            }
        }
    }
\end{algorithm}

\section{兼容性讨论}
\label{sec:rel4_comp}
为了提升用户态程序的易用性,ReL4对seL4的程序提供一定的兼容性。ReL4中已经实现了seL4的基本系统调用并支持对称多处理机(SMP),但采用不同的通知机制和IPC设计以及系统调用处理机制,因此有必要讨论这三部分的兼容性。

\subsection{Notification与U-notification}

相比于原始的通知机制,U-notification在通信权限控制方面主要存在以下两点不同:

\begin{enumerate}
  \item 原始的通知机制允许多个接收线程竞争接收一个内核对象上的通知,这种设计的目的是为了支持多接收端的场景,事实上,多接收端已经可以通过多个内核对象来进行支持,因此这种机制相对冗余,而由于U-notification中接收端对接收线程的独占性,这个能力将不再被支持。
  \item 原始的通知机制允许单个接收线程接收多个内核对象上的通知,这种设计的目的是更灵活地支持多发送端的场景,在U-notification中,同一个内核对象可以被设置为相同的recv status idx,不同的发送端则通过使用中断号(uintr vec)来进行区分。
\end{enumerate}

除了权限控制有所不同之外,改造前后的通信方式也有所区别。原始的通知机制需要用户态接收方通过系统调用主动询问内核是否有通知需要处理。根据是否要将线程阻塞,一般被设计为Wait和Poll两个接口。而U-notification无需接收线程主动陷入并询问内核,接收方被硬件发起的用户态中断打断,并处理到来的通知,这在很大程度上解放了接收方,程序设计者无需关心通知到来的时机,减少了CPU忙等的几率,提升了用户态的并发度。而为了提升U-notification的易用性,ReL4对原始的通信接口进行了兼容:
\begin{enumerate}
  \item Poll: 无需陷入内核态,在用户态读取中断状态寄存器,判断是否有效并返回。
  \item Wait: 对该接口的兼容需要用户态的异步运行时的调度器提供相关支持,在没有有效中断时,该操作将阻塞当前协程并切换到其他协程执行,等待U-notification唤醒。
\end{enumerate}

综上所述,对于多接收方的场景,U-notification可以通过多个内核对象进行实现,除此之外,U-notification可以实现API级别的兼容。

\subsection{同步IPC与异步IPC}

异步IPC通过异步运行时可以在基本通信场景下上实现API级别的兼容,然而seL4中的同步IPC还有额外的能力:
\begin{enumerate}
  \item 错误处理:同步IPC可以用于缺页异常等处理,seL4通过在TCB中维护一个Endpoint对象来发送错误信息给用户态程序进行处理,而在ReL4中,TCB中将维护对应的U-notification对象,以及对应的共享缓冲区指针,当异常和错误发生时,将错误信息写入共享缓冲区,并发送U-notification通知用户态程序。此场景下依然可以实现API级别的兼容。
  \item 能力派生:seL4中的同步IPC拥有能力派生与传递的功能,虽然内核已经支持了Capability Space相关的系统调用,同步IPC使得能力传递更加灵活。而由于异步IPC不经过内核,因此ReL4中不再支持通过IPC来进行能力派生,仅通过系统调用进行能力派生,损失了一部分灵活性,保留了功能的完整性。
\end{enumerate}

此外,异步运行时导致用户态任务模型存在语义上的区别,异步IPC任务将以协程作为任务的基本单位,因此相比于同步IPC任务之外,异步IPC提供了主动让权的API,同时,相比于同步IPC,相同运行时内的不同异步IPC任务不存在并行性,无需同步互斥等操作,提升了用户态易用性。

综上所述,异步IPC在大部分情况下依然能实现API级别的兼容。

\subsection{同步系统调用与异步系统调用}
与异步IPC类似,异步系统调用的发起依然以协程为单位进行,但与同步系统调用的处理不同,为了充分利用CPU硬件,异步系统调用的处理采用多核心、抢占式处理,如果系统中存在空闲核心或执行低优先级任务的核心,异步系统调用会抢占该核心并处理系统调用请求,因此系统调用的发起和处理有可能是多核心且并行的。

此外,有两类系统调用无法异步化:
\begin{enumerate}
  \item 由于异步系统调用依赖于异步运行时,因此与异步运行时初始化相关的系统调用无法被异步化。
  \item 对于实时性要求较高的系统调用无法进行异步化,如$get\_clock()$。
\end{enumerate}

对于上述两类,我们通过异步运行时中的API兼容层进行自动判断,对于无法异步化的系统调用,运行时将自动转化为同步系统调用进行处理。综上所述,异步系统调用可以实现API级别的兼容。

\section{本章小结}
本章系统性地阐述了ReL4微内核的设计实现与兼容性架构。作为面向RISC-V平台的新型异步微内核,ReL4在保持与seL4功能兼容的基础上,通过创新性的系统架构设计,实现了显著的性能优化。

在核心机制方面,ReL4提出了一种基于硬件加速的异步通信框架。通过引入U-notification机制,系统有效降低了传统通知机制带来的上下文切换开销。异步IPC的设计充分利用了现代处理器的多核特性,采用共享缓冲区与协程调度相结合的方式,实现了高效的进程间通信。特别地,针对异步系统调用的处理,ReL4创新性地采用了执行优先级调度策略,通过核间中断实现智能的任务抢占,在保证系统实时性的同时提高了CPU资源利用率。

在兼容性设计上,ReL4展现了良好的系统架构适应性。通过构建多层次的兼容层,系统既保留了seL4原有的API接口语义,又充分发挥了异步架构的性能优势。实验评估表明,这种设计在确保原有应用程序可移植性的前提下,能够显著提升系统的整体性能。后续章节将通过详尽的实验数据,进一步验证这些设计决策的有效性及其对系统性能的影响。
