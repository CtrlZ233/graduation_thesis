%%==================================================
%% abstract.tex for BIT Master Thesis
%% modified by yang yating
%% version: 0.1
%% last update: Dec 25th, 2016
%%==================================================

\begin{abstract}

微内核将大部分系统服务运行在用户空间,相比于宏内核有更好的稳定性、扩展性和内核安全性,在微内核系统中,应用程序通过进程间通信(IPC)而非系统调用来请求服务,频繁的IPC造成的特权级切换将产生巨大开销,成为了系统的性能瓶颈。以seL4为代表的现代微内核将同步IPC作为主要通信手段,并以异步通知机制作为辅助手段来提升系统并发度,这些机制在一定程度上减少了IPC的开销,提升了系统性能,然而它们在设计上仍有三点不足:1)在支持同步IPC的情况下冗余地支持了异步通知机制,这违反了内核最小化原则;2)通知机制依赖内核的转发,会造成大量的特权级切换;3)系统调用和同步IPC会导致无关请求顺序执行,无法充分利用硬件资源。

针对以上三点缺陷,本文设计并实现了ReL4——一套异步化的高性能微内核架构,它在兼容seL4基本系统调用的前提下,将同步IPC从内核中移除,采用纯异步的用户态IPC框架进行通信,保证了内核最小化原则;同时,ReL4基于用户态中断的硬件支持,设计了无需内核转发的异步通知机制(U-notification),避免了通知机制造成的大量特权级切换;最后,ReL4通过异步运行时来实现异步IPC和异步系统调用,避免了无关IPC和系统调用请求的顺序执行,在充分利用硬件资源,提升系统并发度的同时,提升系统易用性,并利用感知任务的中断控制器(TAIC),将任务调度的部分功能卸载到硬件,进一步提升低并发场景的性能。

本文在FPGA上实现了ReL4的原型系统,使用seL4test测试框架证明了ReL4的功能正确性和对seL4的兼容性,同时将U-notification、异步IPC和异步系统调用分别与seL4进行了对比测试,并评估了真实的TCP Server在ReL4的性能表现。测试结果表明,相比于seL4,ReL4能够大幅减少系统中特权级切换的次数,在低并发场景下有着接近的性能,在高并发场景下拥有更卓越的表现,从而证明了ReL4架构的可行性和优越性。

\keywords{微内核; 异步; 进程间通信; 用户态中断}
\end{abstract}  

\begin{englishabstract}
Modern microkernels, which execute most system services in user space, offer superior stability, scalability, and kernel security compared to monolithic kernels. In such systems, applications request services via Inter-Process Communication (IPC) rather than system calls. However, frequent IPC-induced privilege level transitions incur significant overhead, becoming a performance bottleneck. Contemporary microkernels like seL4 address this by adopting synchronous IPC as the primary communication mechanism, supplemented by asynchronous notification mechanisms to enhance concurrency. While these approaches reduce IPC overhead and improve performance to some extent, they still exhibit three key limitations: (1) Redundant support for asynchronous notifications alongside synchronous IPC violates the principle of kernel minimization; (2) Notification mechanisms rely on kernel-mediated forwarding, causing excessive privilege level transitions; (3) System calls and synchronous IPC enforce sequential execution of unrelated requests, underutilizing hardware resources.

To address these deficiencies, we design and implement ReL4—a high-performance asynchronous microkernel architecture. While maintaining compatibility with seL4's essential system calls, ReL4 removes synchronous IPC from the kernel, employing a purely asynchronous user-space IPC framework to uphold the principle of kernel minimization. Additionally, leveraging hardware support for user-space interrupts, ReL4 introduces U-notification, an asynchronous notification mechanism that eliminates kernel-mediated forwarding and the associated privilege level transitions. Furthermore, ReL4 implements an asynchronous runtime to enable concurrent execution of asynchronous IPC and system calls, preventing the serialization of unrelated requests. This design not only enhances resource utilization and concurrency but also improves usability. To further optimize low-concurrency scenarios, ReL4 integrates the Task-Aware Interrupt Controller (TAIC), offloading parts of task scheduling to hardware.
    
We prototype ReL4 on an FPGA platform and validate its functionality and seL4 compatibility using the seL4test framework. Comparative evaluations of U-notification, asynchronous IPC, and asynchronous system calls against seL4 demonstrate significant reductions in privilege level transitions. Performance measurements show that ReL4 achieves comparable performance to seL4 in low-concurrency scenarios and outperforms it in high-concurrency settings. These results confirm the feasibility and superiority of the ReL4 architecture.    

\englishkeywords{microkernel; asynchronous; inter-process communication; user-mode interrupt }

\end{englishabstract}
