%%==================================================
%% abstract.tex for BIT Master Thesis
%% modified by yang yating
%% version: 0.1
%% last update: Dec 25th, 2016
%%==================================================

\begin{abstract}

微内核将大部分系统服务运行在用户空间,相比于宏内核有更好的稳定性、扩展性和内核安全性,在微内核系统中,应用程序通过进程间通信(IPC)而非系统调用来请求服务,频繁的IPC造成的特权级切换将产生巨大开销,成为了系统的性能瓶颈。以seL4为代表的现代微内核将同步IPC作为主要通信手段,并以异步通知机制作为辅助手段来提升系统并发度,这些机制在一定程度上减少了IPC的开销,提升了系统性能,然而它们在设计上仍有三点不足:1)在支持同步IPC的情况下冗余地支持了异步通知机制,这违反了内核最小化原则;2)通知机制依赖内核的转发,会造成大量的特权级切换;3)系统调用和同步IPC会导致无关请求顺序执行,无法充分利用硬件资源。

针对以上三点缺陷,本文设计并实现了ReL4——一套异步化的高性能微内核架构,其主要设计理念是将异步通知机制作为内核支持的唯一通信手段,在用户态通过共享缓冲区实现IPC的数据传递,并通过通知机制进行同步,保证了内核最小化原则;同时ReL4基于用户态中断,在兼容seL4接口的基础上,设计了无需内核转发的U-notification机制,避免了通知机造成的大量特权级切换;最后,ReL4通过异步运行时来实现异步IPC和异步系统调用,避免了无关IPC和系统调用请求的顺序执行,在充分利用硬件资源,提升系统并发度的同时,提升系统易用性。

本文在FPGA上实现了ReL4的原型系统,使用seL4test测试框架证明了ReL4的功能正确性和对seL4的兼容性,同时将U-notification、异步IPC和异步系统调用分别与seL4进行了对比测试,并评估了真实的TCP Server在ReL4的性能表现。测试结果表明,相比于seL4,ReL4能够大幅减少系统中特权级切换的次数,在低并发场景下有着接近的性能,在高并发场景下拥有更卓越的表现,从而证明了ReL4架构的可行性和优越性。

\keywords{微内核; 异步; 进程间通信; 用户态中断}
\end{abstract}  

\begin{englishabstract}
The microkernel runs most system services in the user space. Compared with the monolithic kernel, it has better stability, extensibility, and kernel security. In a microkernel system, applications request services through inter-process communication (IPC) instead of system calls. The frequent privilege level switches caused by IPC generate huge overheads, which become a performance bottleneck of the system. Modern microkernels represented by seL4 use synchronous IPC as the main communication means and adopt an asynchronous notification mechanism as an auxiliary means to improve the system concurrency. These mechanisms reduce the IPC overhead to a certain extent and enhance the system performance. However, there are still three design deficiencies: (1) The asynchronous notification mechanism is redundantly supported while synchronous IPC is supported, which violates the principle of kernel minimization; (2) The notification mechanism relies on the forwarding of the kernel, resulting in a large number of privilege level switches; (3) System calls and synchronous IPC will cause irrelevant requests to be executed sequentially, failing to fully utilize hardware resources.

Aiming at the above three defects, this paper designs and implements ReL4, a set of asynchronous high-performance microkernel architecture. Its main design concept is to use the asynchronous notification mechanism as the only communication means supported by the kernel. In the user space, the data transfer of IPC is realized through a shared buffer, and synchronization is carried out through the notification mechanism, ensuring the principle of kernel minimization. At the same time, based on user-space interrupts and on the basis of being compatible with the seL4 interface, ReL4 designs the U-notification mechanism that does not require kernel forwarding, avoiding a large number of privilege level switches caused by the notification mechanism. Finally, ReL4 implements asynchronous IPC and asynchronous system calls through an asynchronous runtime, avoiding the sequential execution of irrelevant IPC and system call requests. While fully utilizing hardware resources and improving system concurrency, it also enhances the usability of the system.

This paper implements a prototype system of ReL4 on an FPGA. The U-notification, asynchronous IPC, and asynchronous system calls are respectively compared and tested with seL4, and the performance of a real TCP Server on ReL4 is evaluated. The test results show that compared with seL4, ReL4 can significantly reduce the number of privilege level switches in the system, has a similar performance in low-concurrency scenarios, and has more excellent performance in high-concurrency scenarios, thus proving the feasibility and superiority of the ReL4 architecture.

\englishkeywords{microkernel; asynchronous; inter-process communication; user-mode interrupt }

\end{englishabstract}
