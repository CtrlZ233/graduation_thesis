%%==================================================
%% chapter01.tex for BIT Master Thesis
%% modified by yang yating
%% version: 0.1
%% last update: Dec 25th, 2016
%%==================================================
\chapter{总结}
本文针对微内核系统中进程间通信(IPC)的性能瓶颈问题展开研究,提出并实现了基于用户态中断的高性能异步微内核ReL4。通过系统性地重构传统微内核架构,本研究取得了以下创新性成果:首先,提出了一种纯异步的IPC机制,将同步IPC完全移出内核,仅保留异步通知机制,显著简化了内核设计;其次,创新性地利用用户态中断技术设计了U-notification机制,有效减少了特权级切换开销;最后,在用户态设计了异步运行时系统,大幅提升了编程易用性。实验结果表明,该架构使IPC性能提升达3倍,在网络服务器等IPC密集型应用中,系统整体性能最高提升1倍。

本研究提出的异步IPC和异步系统调用机制特别适用于高并发、上下文无关的通信场景。在低并发条件下,通过用户态中断技术和TAIC加速器有效弥补了异步运行时引入的额外开销。测试数据显示,虽然在极低并发场景下性能仍略逊于同步实现,但这一差距随着并发度的提升迅速逆转,足以证明异步微内核架构在性能方面的竞争力。未来工作可考虑通过硬件加速实现异步运行时中的其他关键操作,以进一步消除运行时开销,使系统在各类负载条件下均能保持卓越性能。